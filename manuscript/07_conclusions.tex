
\section{Conclusions}\label{sec5}

The aim of this study was to investigate, comprehend, and correct for RFPA drift. The findings highlight that in typical imaging with a short TR and/or a long RF, RFPA can drift within a notable range. If not addressed, it can manifest as undesired signals, potentially affecting the reproducibility of data. The suggested approach involving predictive and \rev{run}-time drift correction has shown encouraging results in mitigating drift. This involves monitoring DICO recordings to calculate drift and dynamically adjusting transmit voltages during the scanning. The suggested drift correction method also has the potential to enhance the accuracy and reproducibility of qMRI performed on an MR system equipped with unregulated amplifiers.

% \mycounter{\quickwordcount{07_conclusions}}