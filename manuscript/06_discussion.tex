
\section{Discussion}\label{sec4}

In this study, we conducted an evaluation of RFPA drift and examined several contributing factors. With the aim of real-time measurements of forward/reflected signal through DICOs, the extent of drift can be quantified during any scanning. Here, we utilized DICO feedbacks and introduced two approaches for prospective corrections of RFPA drift. \marginnote{R1.1}\rev{Utilizing RF current and voltage sensors\cite{stang2009versatile}, pick-up coils\cite{graesslin2013detection, graesslin2015comprehensive}, gradient reversal approach to evaluate RF (GRATER)\cite{landes2020iterative}, and magnetic field monitoring using NMR sensors\cite{brunner2016concurrent} are alternative methods proposed in the literature for RF monitoring and potentially for measuring RFPA drift. When compared to DICOs, pick-up coils may offer certain advantages, such as providing additional information that might not be necessary for drift correction and offering a more direct approach. However, incorporating pick-up coils requires modifications for each RF coil, whereas DICOs are already integrated into the transmission pipeline, making them a more convenient choice. }

A typical initial drift curve for a train of short RFs \marginnote{R1.15}\rev{appears to align with a two-term power series model expressed as $ax^{b}+c$. Here, the variables a, b, and c represent variable coefficients, while x denotes the drifting samples.} However, it is understood from the assessment study that the drift evolution varies from one protocol to another. Furthermore, in instances where hardware designs are not fully matched, as in this work, the drift behavior of RFPAs can exhibit discrepancies. Hence, in addition to the two presented methods for drift correction, the utilization of an empirical model incorporating relevant protocol parameters could potentially offer a prediction of drift, even though this aspect was not explored in this study.

RFPAs are susceptible to voltage droop regardless of the transmit voltage. This is because of the rise in the temperature of the RF transistors which which often leads to a decrease in gain.\cite{Myer2011} \marginnote{R1.16}\rev{However, it is highly plausible that the circuitry of the RFPA employed in this study includes capacitor banks intended to discharge and counteract the droop effect. \marginnote{R2.6}The effectiveness of these capacitors can be dictated by the energy they store and the rate at which they discharge make them optimal for specific operational range, potentially leading to overcompensation or undercompensation in other voltage ranges. Both RF transmit voltage and pulse duration are significant factors. It is expected the droop effect becoming noticeable at lower voltages when the RF pulse duration is getting longer. It is plausible that} at lower transmit voltages \rev{for a regular short RF pulse}, an overcompensation for RFPA output leads to an observable increase in drift, as depicted by the upward trend in Figure \ref{fig:drift_assessment1}. The experiments which are not employing a high RF power -- which is the case in the imaging protocols of this study and wide range of protocols -- were observed to be immune against voltage droop and did not experience a significant phase drift. Consequently, the corrections were exclusively computed based on the magnitude of DICOs records in this work.

\marginnote{R1.2}\rev{The inter-scanner assessment of inter-pulse drift indicates a similar trend between 9.4T and 7T Plus scanners, while 11.7T and 7T Terra exhibit a distinct pattern. At 11.7T and 7T Terra, the drift curve shifts direction, beginning to decrease after reaching to its peak. This results in prolonged drifting and requires more time for RFPAs to settle. This difference might be attributed to the design variation, with 2\,kW amplifiers used in 11.7T and 7T Terra scanners compared to the 1\,kW amplifiers used in 9.4T and 7T Plus scanners. Separate evaluation of the RFPA drifting behavior at each site is recommended in light of these observations. The assessment of drift was conducted multiple times using different RF coils and loads (phantoms and subjects). Additionally, at one site, the measurement was repeated while the scanner was ramped down. Overall, it was observed that the aforementioned factors do not contribute to RFPA drift, or their contribution is minimal. Although all experiments in this study were conducted on UHF MR scanners, the observation of RFPA drift at lower fields with quadrature body coils is not unexpected.}

This study investigated two k-space filling strategies: linear and centric. While linear ordering is the more common approach in Cartesian sampling, the centric reordering of k-space patterns is commonly preferred for sequences involving magnetization preparation or when the transient signal is of interest. \cite{bosch2023optimized, zaiss2018snapshot, worters2010balanced} The steeper initial slope of the drift curve can result in a more noticeable change in \marginnote{R1.18}\rev{global }intensity in the image domain for centrically ordered acquisitions. This effect is evident through the higher CV observed in Figure \ref{fig:phantom_drift_result} for the centrically ordered scan compared to the linearly ordered scan.


The predictive approach offers a lower implementation complexity. Unlike the \rev{run}-time approach, it does not demand a heavy computational burden for real-time calculations, alteration of the reconstruction pipeline, and the integration of feedback mechanisms. However, this method necessitates the repetition of the entire protocol once to generate the RFPA drift curve, followed by offline calculations to determine the necessary corrections. This approach relies on the premise that the drift curve remains consistent when the same protocol is repeated, an assumption that has been verified as valid \marginnote{R2.9}\rev{for the protocols utilized in this study. Supporting Information Figure \ref{sif:drift_repeatability} exhibits a demonstration of the reproducibility of drift curves involving five measurements with a 10-second interval between each (TR=3.5ms, RF=1ms). In typical protocols, exporting raw-data and computing scale factors with predictive correction approach provides ample time for RFPA for complete relaxation. Nevertheless, in exceptionally high RF duty-cycle scenarios,\marginnote{R1.10} it is recommended to contemplate extending this time period for a full relaxation and experimentally verify it for individual RFPAs.} 


When the transmit voltage remains constant, the correction table derived from one scan can be employed for the same scan in different sessions. However, it is important to note that in this study, slight instability was observed in certain RFPAs, which challenges the assumption of drift curve reproducibility. RFPAs exhibiting instability are those that deviate from the overall curve trend depicted in Figure \ref{fig:drift_assessment1}. This instability is further evident in the predictive correction DICO plots displayed in Figure \ref{fig:sub3_drift_plot}, where one RFPA, plotted in green, demonstrates this behavior. To address this concern, the \rev{run}-time correction approach can be employed as a solution. The \rev{run}-time approach eliminates the need for repeated scans as the correction factors are computed directly from the ongoing scan. This capability enables the cancellation of both drift and unexpected RFPA instability. However, it is important to account for the time lag between calculations and the application of correction factors. This lag can extend to several TR when TR is very short. In the case of a TR of 3.2\,ms, as utilized in this study \rev{for bSSFP sequence}, the correction calculated for the n$^\text{th}$ pulse was applied to either the (n+2)$^\text{th}$ or (n+3)$^\text{th}$ pulse due to the time lag. The extent of lag is influenced by factors such as the RF length, the number of transmit channels, and the TR. In cases where the TR exceeded 15\,ms and the RF length was 1\,ms, it became feasible to reduce the delay to just one TR.


To achieve a fully established steady state, it is advisable to utilize a preparation scan of 5T$_1$\cite{bieri2013fundamentals}. Because of the extended T$_1$ values of GM and WM at 9.4T\cite{zhu2014relaxation}, the transition to steady state requires approximately 10 seconds, during which RFPAs reach a stable output condition. In order to maintain the sensitivity to RFPA drift, only the initial seven repetitions were excluded for in-vivo analysis. This determination was made based on the intensity curve of a representative voxel, plotted in the final column of Figure \ref{fig:sub3_drift_result}. This means that a total of 1136 RF pulses were employed as part of the preparation scan, with an average drift contribution of only 2.0\% factoring into the CV calculation. Nevertheless, the variability in voxel intensity across the initial repetitions remains a combination of RFPA drift and the transient state of the bSSFP sequence. Disentangling the individual contributions from these two factors proves to be a challenging task. Simulation results indicated that the utilization of a low FA pulse could expedite the attainment of steady-state. As demonstrated in Figure \ref{fig:drift_assessment3}, the reduction in FA should not impact the maximum drift, provided the study remains within a low transmit voltage range. However, it has been illustrated in Supporting Information Figure \ref{sif:signal_drift_sensitivity} that for a constant RFPA drift, the MR signal exhibits decreased sensitivity to RFPA drift at lower FA values. To validate the proposed RFPA drift correction method, a custom-made spherical gel phantom with a $\text{T}_1$ relaxation time of 460\,ms was employed. This choice enabled a notable reduction in the steady-state transition time.

\subsection{Potential benefit of RFPA drift correction}

The sensitivity to RFPA drift varies based on the specific application. In certain cases, the impact of drift can be mitigated by running dummy scans, allowing the RFPA output to stabilize. However, this approach may not always be feasible. Particularly in UHF scenarios where $\mathrm{B_{1}^{+}}$ inhomogeneity requires the use of parallel transmission, both $\mathrm{B_{1}^{+}}$ mapping calibration scans and tailored RF pulses can be influenced by RFPA drift. This can raise disparity between simulations and actual measurements and needs to be considered. The unaccounted impact of RFPA drift could potentially compromise the accuracy and safety of SAR predictions. RFPAs with different magnitude/phase drifts can cause the SAR distribution to deviate from the expected distribution. Even if all RFPAs drift consistently with each other, the magnitude of the SAR will be overestimated or underestimated. While the scanner used in this study uses real-time SAR monitoring to measure power for all 16 channels, SAR lookahead is considered a redundancy to increase the fault tolerance of the SAR monitoring system, which is compromised by drifting RFPAs. In addition, underestimation of SAR by the prospective SAR supervision mechanism can cause nuisance if sequences are aborted by online SAR supervision if the two deviate due to RFPA drift.

Considering the long-standing aim of quantitative MRI to establish a common ground for quantitative tissue characteristics, which is ideally completely independent of any external instrumental factors and thus provides an objective reproducible measure, an understanding of the occurrence of RFPA drift and its correction is crucial. The results presented in this article indicate that RFPA drift could potentially affect the accuracy of any MR quantification method, which is sensitive to the RF excitation profile and flip angle, in particular in scenarios with high RF duty cycles and when using unregulated RF amplifiers. Candidates which could potentially benefit from an RFPA drift analysis and correction strategy include CEST and magnetization transfer experiments \cite{VanZijl2018}, fingerprinting \cite{Ma2017}, \rev{spin tomography in the time domain} \cite{Sbrizzi2018}\marginnote{R1.20}, short-TR steady-state relaxometry \cite{Deoni2007, weiskopf2013quantitative, Nguyen2017, Shcherbakova2018, Heule2014}, or $\mathrm{B_{1}^{+}}$ mapping sequences such as \marginnote{R2.1}\rev{the investigated satTFL} or BSS \cite{Sacolick2010}. 

\marginnote{R2.1}\rev{The potential impact on quantitative MRI is corroborated by the performed $\mathrm{B_{1}^{+}}$ mapping experiments using satTFL, which reveal that RFPA drift affects the reliability of the obtained $\mathrm{B_{1}^{+}}$ values (cf. Figure \ref{fig:tflb1}). Errors in the estimation of $\mathrm{B_{1}^{+}}$ can in turn considerably affect the quantification of tissue-specific parameters such as relaxation or diffusion metrics if the underlying MR acquisition is sensitive to $\mathrm{B_{1}^{+}}$, which is, for example, the case for SSFP sequences. The simulation results presented in Supporting Information Figure \ref{sif:T1_B1_dep} demonstrate for three SSFP-based $\mathrm{T_{1}}$ mapping techniques—variable flip angle (VFA)\cite{heule2016variable}, motion-insensitive rapid configuration relaxometry (MIRACLE)\cite{Nguyen2017}, and triple echo steady state (TESS)\cite{Heule2014}—how errors in $\mathrm{B_{1}^{+}}$ mapping can translate into a bias of $\mathrm{T_{1}}$ estimation since those methods necessitate knowledge about $\mathrm{B_{1}^{+}}$, which is typically obtained by an external $\mathrm{B_{1}^{+}}$ mapping scan. The simulation reveals that a deviation of approximately 5\% in $\mathrm{B_{1}^{+}}$ leads to a corresponding deviation of about 10\% in $\mathrm{T_{1}}$ for all investigated methods. As evident from Figure \ref{fig:tflb1}, RFPA drift can cause $\mathrm{B_{1}^{+}}$ deviations exceeding 5\% in large portions of the brain and as high as 15\% in certain areas. Please note that MIRACLE and TESS allow the joint quantification of $\mathrm{T_{2}}$ alongside $\mathrm{T_{1}}$. In contrast to $\mathrm{T_{1}}$, the $\mathrm{T_{2}}$ estimation is largely independent of $\mathrm{B_{1}^{+}}$ for these methods \cite{Ganter2023}.} Generally, RFPA drift is expected to become the more apparent the shorter the scan times and should thus be considered especially in case of accelerated quantitative imaging with sparsely sampled data.

\subsection{Limitations}

In the suggested RFPA drift correction approach, the reference is established using the integral of the magnitude of the initial RF pulse, with subsequent pulses scaled relative to this reference. This implies that the correction focuses solely on inter-pulse variations, striving to maintain a consistent RF integral. However, it is important to note that drift also occurs while the RF is actively transmitting, leading to modifications in the RF shape. For example, a rectangular RF pulse can be experienced as a right trapezoid, potentially affecting the excitation profile. This particular effect has not been explored or considered in the current work.


% \mycounter{\quickwordcount{06_discussion}}