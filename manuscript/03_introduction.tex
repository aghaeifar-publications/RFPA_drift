\section{Introduction}\label{sec1}

Magnetic Resonance Imaging (MRI) relies on the precise functioning of complex hardware components. While MRI technology has advanced significantly over the years, one challenge that continues to persist is the occurrence of drift (gradual deviation in the performance over time) in the MRI hardware. Drift in MRI hardware can result in reduced image quality and decreased reproducibility of imaging studies over time, compromising the accuracy and diagnostic value of the images.

The severity and impact of various hardware drift types on the acquisition process are determined by the employed sequence and utilized protocol. For instance, in MR imaging, drift of the $B_0$ field can result in image shift along the phase-encoding (PE) direction in echo-planar imaging (EPI) \cite{thesen2003absolute} or ineffective fat saturation \cite{Benner2006}. In MR spectroscopy, correcting for frequency and phase drift is an essential step as spectral broadening and SNR loss will impact metabolite signals if not accounted for.\cite{Near2015, Steve2021, Lange2011}

\rev{Another element within MRI known to undergo drift is the RF power amplifier (RFPA). This component is tasked with amplifying the RF pulse in the sequence, and its abnormality in performance results in varying excitation profiles and flip-angles.\cite{Myer2011}} To compensate, feedback routines can be integrated at the hardware level. \marginnote{R1.1}\rev{It is shown that employing current feedback within an on-coil class-D RFPA system enables obtaining RF envelope information in terms of amplitude and bandwidth, thereby achieving load-insensitive operation.\cite{Gudino2013} Another work demonstrates the potential application of Cartesian feedback to control in-phase and quadrature components of the current delivered to a RF coil.\cite{hoult2008overcoming} This method was subsequently expanded to integrate a band-pass error amplifier, addressing sensitivity issues related to phase or amplitude discrepancies as well as DC offsets in the feedback loop.\cite{Zanchi2011}} However, there are instances where corrections can be delayed or compromised during specific acquisition protocols where RFPA undergo high thermal stress when they are subjected to high duty cycle tasks. 

\marginnote{R1.1}\rev{Correction of RF pulse imperfections can be expanded to software level, offering increased reproducibility and flexibility for the development and testing of algorithms. A study shows RF envelope can be adjusted iteratively using basic signal addition and subtraction to achieve a desired excitation profile without requiring compensation for nonlinearities in the RFPA.\cite{lebsack2002iterative} Another approach involves utilizing a current sensor\cite{stang2009versatile} and supplying the RFPA with pre-distorted input, aiming to rectify offset, non-linearity, coupling, and temporal errors.\cite{stang2009vector} The pre-distortion technique can also be implemented using data gathered from field sensors.\cite{ccavucsouglu2017correction} In the context of parallel excitation pulses where a rapidly changing envelope is typical, a regularization technique has been proposed to mitigate distortions arising from the memory effect of the RFPA.\cite{grissom2010minimum} Despite not being currently utilized, these methods hold potential for compensating RFPA drift with certain modifications. However, they depend on external hardware, iterative processes, or calibration scans for their implementation.}

\rev{At ultra-high field (UHF) MRI, this issue becomes notably more troublesome as greater power requirements arise, often necessitating RF power up to 10-32 kW.\cite{Myer2011,hoult2008overcoming}}. Consideration should also be given the requirement of multiple RFPAs needed for RF transmit arrays which are in most cases preferred to volume body coils at UHF in order to mitigate $B_1^+$ inhomogeneity.\cite{IBRAHIM2000,feng201264} Multiple RFPAs outputs can be combined and later split or individually utilized to drive a pTx array coil. The output drift may vary between RFPA revisions (model numbers), which can be problematic when malfunctioning units are replaced with newer models with an updated hardware design. Additionally, differences in RF waveforms played out in individual channels, often associated with tailored pTx RF pulses, can also contribute to \marginnote{R1.7}\rev{different drift behaviors as the RFPA transfer function does not necessarily offer a constant power gain factor \cite{Myer2011}, and the drifting behavior might also exhibit variations under varying stress conditions.}

RFPA drift is a critical aspect to consider to obtain comparable results in inter-site and intra-site reproducibility studies. The impact of RFPA drift on quantitative imaging was recently investigated as part of the traveling head 2.0 multi-center study at 7T. \cite{VOELKER2021117910} It was observed that the RFPA behaviour differed across sites. Despite an average deviation of 6.3\% in the measured $B_1^+$ between the individual UHF sites with different hardware and software versions, comparable results were obtained. However, by employing calibration data to correct for RFPA drift retrospectively, the observed \marginnote{R1.8}\rev{initial differences between the different centers} could be reduced by 8.1\% for CEST relayed nuclear Overhauser effect (rNOE) maps. The repeatability of high-resolution multi-parameter mapping (MPM) was evaluated in a separate multi-site study conducted at 7T.\cite{sherif2022repeatability} All five sites had a scanner equipped with identical hardware and configuration (which was not the case for the traveling head study). While MPM measurements generally exhibited a good level of repeatability, consistently across all sites, the magnetization transfer saturation (MTsat) map exhibited the highest coefficient of variation (CV), indicating lower repeatability. Furthermore, using the Bloch-Siegert shift (BSS)\cite{Sacolick2010} $B_1^+$ mapping method for transmit inhomogeneity correction, the $R_1$ and PD maps demonstrated slightly higher CV. Given the utilization of RF pulses with extended duty cycles in CEST, BSS, and MTsat sequences, we suspect that RFPA drift might be a principal factor underlying the reduced repeatability observed in the MTsat, $R_1$, and PD maps in\cite{sherif2022repeatability}. This reasoning gains credence from the fact that the scanners utilized in these investigations are outfitted with unregulated amplifiers, which are not specifically tailored for quantitative MRI (qMRI) applications.

In this study, our initial focus is to examine the behaviour of RFPA under specific protocols in order to evaluate the influences of various factors and dependencies on RFPA drift, as well as their reproducibility. We provide evidence of RFPA drift based on phantom and in vivo scans and compare the drift-affected data with their corresponding drift-compensated counterparts. We demonstrate the utilization of calibration data for prospective drift correction. Furthermore, we propose a method to monitor the output of RFPA and effectively correct for any potential deviations by providing \rev{run}-time feedback to the imaging sequence. Lastly, we discuss some strategies aimed at reducing impact of RFPA drift in the reconstructed data.


% \mycounter{\quickwordcount{03_introduction}}