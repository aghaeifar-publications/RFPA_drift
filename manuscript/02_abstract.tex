
\abstract{
\section{Purpose} The drift in RF power amplifiers (RFPA) is assessed and several contributing factors are investigated. Two approaches for prospective correction of drift are proposed and their effectiveness is evaluated.   
\section{Methods} RFPA drift assessment encompasses both intra-pulse and inter-pulse drift analyses. Scan protocols with varying FA, RF length, and TR are used to gauge the influence of these parameters on drift. Directional couplers (DICOs) monitor the forward waveforms of the RFPA outputs. DICOs data is stored for evaluation, allowing calculation of correction factors to adjust RFPAs' transmit voltage. Two correction methods, predictive and \rev{run}-time, are employed: predictive correction necessitates a calibration scan, while \rev{run}-time correction calculates factors during the ongoing scan.  
\section{Results} RFPA drift is indeed influenced by the RF duty-cycle, and in the cases examined with a maximum duty-cycle of 66\%, the potential drift is approximately 41\% or 15\%, depending on the specific RFPA revision. Notably, in low transmit voltage scenarios, FA has minimal impact on RFPA drift. The application of predictive and \rev{run}-time drift correction techniques effectively reduces the average drift from 10.0\% to less than 1\%, resulting in enhanced MR signal stability.

\section{Conclusion} Utilizing DICO recordings and implementing a feedback mechanism enable the prospective correction of RFPA drift. Having a calibration scan, predictive correction can be utilized with fewer complexity; for enhanced performance, a \rev{run}-time approach can be employed. }

\keywords{RF power amplifier, Drift, Directional Coupler, bSSFP, \rev{run}-time correction, satTFL}


% \jnlcitation{\cname{%
% \author{Williams K.}, 
% \author{B. Hoskins}, 
% \author{R. Lee}, 
% \author{G. Masato}, and 
% \author{T. Woollings}} (\cyear{2016}), 
% \ctitle{A regime analysis of Atlantic winter jet variability applied to evaluate HadGEM3-GC2}, \cjournal{Magn. Reson. Med.}, \cvol{2017;00:1--6}.}